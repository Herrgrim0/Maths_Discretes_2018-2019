\documentclass[fontsize=10pt]{article}
\usepackage[utf8]{inputenc}
\usepackage[T1]{fontenc}
\usepackage{graphicx} % handles figures
\usepackage[fleqn]{mathtools}
\usepackage{amsmath}
\usepackage{hyperref}
\usepackage{amssymb}
\usepackage[makeroom]{cancel}

%Insertion de tout un tas de librairie qui nous seront probablement inutiles pour la pluspart mais it's always good to have them
\title{\textbf{TP1}\\ Solutions}
\author{Beltus Marcel}
\date{}
\begin{document}
\maketitle % fais le titre écris plus haut


\section*{Exercice 1}
\begin{enumerate}
\item $|A\times B | = |A|\cdot|B|=ab$
\item $|B^A|=b^a$
\item $|\{f:A \rightarrow B : \text{f est une projection de A dans B, injective}\}| = \frac{b!}{(b-a)!}\text{si $a \leq b$, sinon 0.}$
\item $|S(A)| = a!$
\end{enumerate}
\section*{Exercice 2}
\begin{enumerate}
\item $|F^x| = 1$, donc $|F|=1$ puisque $|F^x|=f^x$ pour $|F|=f$ et $|X|=x$ .
\item $|Y^F|=1$, donc F n'existe pas si non vide.
\end{enumerate}

\section*{Exercice 3}
\begin{enumerate}
\item \begin{align*}
\text{Supposons $f$ non-injective}&\\
&\text{alors } \exists a,b \in A \text{ tq } f(a)=f(b)\\
&\text{et donc } g\mathord\circ f (a) = g\mathord\circ f (b)\\
&\text{mais }g\mathord\circ f \text{ est injective, donc $f$ doit être inejctive}
\end{align*}
\item \begin{align*}
\text{Supposons $g$ non surjective}& \\
&\text{alors } \exists c \in C : \nexists b \in B \text{ tq } g(b) = c\\
&\text{et donc }\exists c \in C : \nexists a \in A \text{ tqt } g\mathord\circ f(a ) = c\\
&\text{mais $g\mathord\circ f$ est surjective, donc $g$ doit être surjective}
\end{align*}
\item \begin{align*}
\text{si $g\mathord\circ f$ est bijective, alors }&\text{$g\mathord\circ f$ est surjective et injective}\\
& \text{comme démontré en 1, $f$ est injective}\\
& \text{comme démontré en 2, $g$ est donc sujective}
\end{align*}
\end{enumerate}
\section*{Exercice 4}
Explications not relevant, to add.
\section*{Exercice 5}
$\mathbb{N}=\{ 0,1,2,\dots,m-1,m\}$ et $\mathbb{Z} = \{0,1,-1,2,-2,\dots,n,-n\}$
\[
  f:\mathbb{N}\rightarrow\mathbb{Z} =\begin{cases}
               n \rightarrow \frac{-n}{2} \hspace{1cm}\text{ si $n$ pair}\\
               \hspace{7cm}\text{(injection)}\\
               n \rightarrow \frac{n+1}{2} \hspace{1cm} \text{ si $n$ est impair}
            \end{cases}
\]

\[
  g=f^{-1}:\mathbb{Z}\rightarrow\mathbb{N} =\begin{cases}
               n \rightarrow -2n \hspace{1cm}\text{ si $n\leq 0$}\\
               \hspace{7cm}\text{(surjection)}\\
               n \rightarrow 2n-1 \hspace{1cm}\text{ si n>0}
            \end{cases}
\]
\section*{Exercice 6}
si $n$ est pair, alos $n^3$ est pair, donc $n^3-n$ est pair.\\
si $n$ est impair, alors $n^3$ est impair, donc $n^3 -n$ est pair. \hspace{0.5cm}(modulo 2 si poussé)
\section*{Exercice 7}
\begin{align*}
\text{supposons } \sqrt{3}\in \mathbb{Q}&\\
&\text{alors }\sqrt{3} = \frac{a}{b} \text{ , } a, b \in \mathbb{Z} \text{ et } b\neq 0\\
&\text{alors }\sqrt{3}=\frac{3}{\sqrt{3}} \text{ , et } a =3, b = \sqrt{3}\\
&\text{mais } b \notin \mathbb{Z} \text{ quand } b =\sqrt{3}\text{, donc } \sqrt{3} \notin \mathbb{Q} 
\end{align*}
\section*{Exercice 8}
\begin{align*}
18a+6&b=1\\
3a+&b=\frac{1}{6}\\
&b = \frac{1}{6} - 3a
\end{align*}
Supposons $b\in \mathbb{Z}$, alors $a\notin \mathbb{Z}$ et inversément
\end{document}
