\documentclass[fontsize=10pt]{article}
\usepackage[utf8]{inputenc}
\usepackage[T1]{fontenc}
\usepackage{graphicx} % handles figures
\usepackage[fleqn]{mathtools}
\usepackage{amsmath}
\usepackage{hyperref}
\newcommand\tab[1][1cm]{\hspace*{#1}}
\usepackage{graphicx}

%Insertion de tout un tas de librairie qui nous seront probablement inutiles pour la pluspart mais it's always good to have them
\title{\textbf{Séance 7}}
\author{Antunes André}
\date{}
\begin{document}
\maketitle % fais le titre écris plus haut

\section{Exercice 1}

cas 1 : 1 marche, reste n-1 \\
cas 2 : 2 marches, reste n-2 \\
\tab pour $a_{n}$, le nombre de marches à monter, \\
\tab \tab $a_{n} = a_{n-1}+a_{n-2}$ \\
\tab \tab $F_{n+1}$ (et pas $F_{n}$ car pas de répétition de 1)\\
\tab \tab donc pour n=30, $F_{31}$

\section{Exercice 2}
\tab \tab $D_{n} = D_{n-1}+D_{n-2}$ ou $F_{n+1}$ \\

\section{Exercice 3}
 $$\lim_{n\to\infty} \frac{F_{n+1}}{F_n{}} = \lim_{n\to\infty} \frac{\frac{1}{\sqrt{5}}[\varphi^{n+1}-\widetilde{\varphi}^{n+1}]}{{\frac{1}{\sqrt{5}}[\varphi^{n}-\widetilde{\varphi}^{n}]}} = $$ \\
 --> voir avec Caroline ,exo manquante
 
\clearpage
 
\section{Exercice 4}
\includegraphics[width=\textwidth]{4}

\section{Exercice 5}
\includegraphics[width=\textwidth]{5}

\section{Exercice 6}
\includegraphics[width=\textwidth]{6-1}
\includegraphics[width=\textwidth]{6-2}

\section{Exercice 7}
\includegraphics[width=\textwidth]{7}

\section{Exercice 8}
\includegraphics[width=\textwidth]{8}

\section{Exercice 9}
\includegraphics[width=\textwidth]{9-1}
\includegraphics[width=\textwidth]{9-2}


\end{document}
