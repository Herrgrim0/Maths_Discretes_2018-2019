\documentclass[fontsize=10pt]{article}
\usepackage[utf8]{inputenc}
\usepackage[T1]{fontenc}
\usepackage{graphicx} % handles figures
\usepackage[fleqn]{mathtools}
\usepackage{amsmath}
\usepackage{hyperref}
\newcommand\tab[1][1cm]{\hspace*{#1}}
\usepackage{graphicx}

%Insertion de tout un tas de librairie qui nous seront probablement inutiles pour la pluspart mais it's always good to have them
\title{\textbf{Séance 7}}
\author{Antunes André}
\date{}
\begin{document}
\maketitle % fais le titre écris plus haut

\section*{Exercice 1}

cas 1 : 1 marche, reste n-1 \\
cas 2 : 2 marches, reste n-2 \\
\tab pour $a_{n}$, le nombre de marches à monter, \\
\tab \tab $a_{n} = a_{n-1}+a_{n-2}$ \\
\tab \tab $F_{n+1}$ (et pas $F_{n}$ car pas de répétition de 1)\\
\tab \tab donc pour n=30, $F_{31}$

\section*{Exercice 2}
\tab \tab $D_{n} = D_{n-1}+D_{n-2}$ ou $F_{n+1}$ \\

\section*{Exercice 3}
Prenons $\alpha = \lim_{n\to\infty} \frac{F_{n+1}}{F_n}$
\begin{align*}
\alpha &= \lim_{n\to\infty} \frac{F_n + F_{n-1}}{F_n}\\
&= 1 + \lim_{n\to\infty} \frac{F_{n-1}}{F_n}\\
&= 1 + \lim_{n\to\infty} \frac{F_{n}}{F_{n+1}}\\
&= 1 + \frac{1}{\lim_{n\to\infty} \frac{F_{n+1}}{F_{n}}}\\
&= 1 + \frac{1}{\alpha}\\
&= \frac{1 \pm \sqrt{5}}{2}\\
&= \phi
\end{align*}
 
\section*{Exercice 4}
\begin{align*}
& n=1, \phantom{a}\phi^1 = 1 \times \phi + 0\\
& n=2, \phantom{a}\phi^2 = 1 \times \phi + 1\\
& n=n+1, \phantom{a}\phi^{n+1} = (F_n + F_{n-1})\phi + F_n\\
& \phantom{aaaaaaaaaaaaaa}= F_n (\phi + 1) + \phi F_{n-1}\\
& \phantom{aaaaaaaaaaaa}\phi^n = F_n\phi + F_{n-1}
\end{align*}

\section*{Exercice 5}
\begin{align*}
& n=3, \phantom{a}2 > \phi \\
& n=n-1, \phantom{a}\phi^2 F_{n-1} > \phi^{n-1}\\
& \phantom{aaaaa} (\phi + 1) F_{n-1} > F_{n-1} \phi + F_{n-2}\\
& \phantom{aaaaaaaaaaa} F_{n-1} > F_{n-2}\\
& n=n+1, \phantom{a}F_n + F_{n-1} > F_{n-1} \phi + F_{n-2}\\
& \phantom{aaaaaaaa} F_{n-1} + F_{n-2} > F_{n-1} (\phi - 1) + F_{n-2}\\
& \phantom{aaaaaaaaaaaaaa} 2 F_{n-1} > F_{n-1} \phi
\end{align*}

\section*{Exercice 6}
\textit{Si vous avez le moindre doute sur les exos suivants, Wolfram Alpha est votre ami.}
\begin{enumerate}
\item $a_n - \frac{1}{2}a_{n-1} = 1$
\begin{align*}
& \text{EHA: } x - \frac{1}{2} = 0, \text{ donc solution: }C\frac{1}{2}^n\\
& \text{si } \tilde{a}_n = A, \text{ alors } A = \frac{1}{2}A + 1, \text{ soit } A = 2\\
& \text{donc } a_n = 2 + C\frac{1}{2}^n.
\end{align*}
Sachant que $a_0 = 1$, $2 + C = 1$, $C = -1$.\\
Donc $a_n = 2 - \frac{1}{2}^n$.

\item $a_n - 5a_{n-1} + 6a_{n-2} = 0$
\begin{align*}
& \text{EHA: } x^2 - 5x + 6 = 0, \text{ donc solution: }A2^n + B3^n\\
& \text{donc } a_n = A2^n + B3^n.
\end{align*}
Sachant que $a_0 = -1$, $A+B = -1$.\\
Sachant que $a_1 = 1$, $2A+3B = 1$, $A=-4$ et $B=3$.\\
Donc $a_n = 3^{n+1} - 2^{n+2}$.

\item $a_n - 6a_{n-1} + 9a_{n-2} = 0$
\begin{align*}
& \text{EHA: } x^2 - 6x + 9 = 0, \text{ donc solution: } (An + B)3^n\\
& \text{donc } a_n = (An + B)3^n.
\end{align*}
Sachant que $a_0 = 1$, $B = 1$.\\
Sachant que $a_1 = 9$, $(A+1)3 = 9$, $A=2$.\\
Donc $a_n = (2n + 1)3^n$.

\item $a_n - 4a_{n-1} + 3a_{n-2} = 2^n$
\begin{align*}
& \text{EHA: } x^2 - 4x + 3 = 0, \text{ donc solution: } A3^n + B1^n\\
& \text{si } \tilde{a}_n = C2^n, \text{ alors } C2^n = 4C2^{n-1} - 3C2^{n-2} + 2^n\\
& \phantom{aaaaaaaaaaaaa} 4C2^{n-2} = 8C2^{n-2} - 3C2^{n-2} + 4(2^{n-2})\\
& \phantom{aaaaaaaaaaaaaaaaaa} C = -4\\
& \text{donc } a_n = A3^n + B - 2^{n+2}.
\end{align*}
Sachant que $a_0 = 1$, $A+B-4 = 1$, $A+B = 5$.\\
Sachant que $a_1 = 11$, $3A+B-8 = 11$, $A = 7$ et $B = -2$.\\
Donc $a_n = 7(3^n)-2^{n+2} - 2$.

\end{enumerate}

\section*{Exercice 7}
\begin{enumerate}
\item $a_n = \frac{1}{5} \left[4^{n+1} + (-1)^n \right]$\\
solution générale: $a_n = A4^n + B(-1)^n$
\item $a_n = 5 - 2^{n+2} + 3^n$\\
solution générale: $a_n = A1^n + B2^n + C3^n$
\item $a_n = \frac{1}{9} \left[8 - 6n + (-2)^n\right]$\\
solution générale: $a_n = A(-2)^n + (Bn + C)$
\item $a_n = An^2 + B^n + C$
\item $a_n = 2^{-n/2}(Ai^n + B(-1)^n + C(-i)^n + D)$\\
\textit{Racine quadruple imaginaire, bon amusement, sans moi, merci bien.}
\end{enumerate}

\section*{Exercice 8}
$a_{n+2} - 2Cos(\alpha)a_{n+1} + a_n = 0$\\\\
Petit rappel:\\
$Cos(\alpha) = \frac{1}{2}(e^{i\alpha} + e^{-i\alpha})$\\
\begin{align*}
& \text{EHA: } x^2 - (e^{i\alpha} + e^{-i\alpha})x + 1 = 0, \text{ donc solution: } Ae^{i\alpha n} + Be^{-i\alpha n}\\
& \text{donc } a_n = Ae^{i\alpha n} + Be^{-i\alpha n}.
\end{align*}
Sachant que $a_1 = Cos(\alpha)$, $Ae^{i\alpha} + Be^{-i\alpha} = \frac{1}{2}(e^{i\alpha} + e^{-i\alpha})$, $A=B=\frac{1}{2}$.\\
Donc $a_n = Cos(\alpha n)$.

\section*{Exercice 9}
\begin{enumerate}
\item $a_n = \frac{1}{6} \left[7(-2)^n + 2n + 11 \right]$\\
solution générale: $a_n = \frac{1}{6}\left[2n + 11\right] + A(-2)^n$\\
$\tilde{a}_n = Bn + C$
\item $a_n = 3^n + (-9)^n$\\
solution générale: $a_n = 3^n + A1^n + B(-9)^n$\\
$\tilde{a}_n = C3^n$
\item $a_n = 2^n + \frac{1}{4}(n-1) + (An + B)3^n$\\
$\tilde{a}_n = C2^n + Dn + E$
\item $na_n = (n+3)a_{n-1} + n^2 + n$
\begin{align*}
\text{EHA: } na_n &= (n+3)a_{n-1}\\
a_n &= \frac{1}{n}(n+3)a_{n-1}\\
&= \left(\underset{i=1}{\overset{n}{\prod}} \frac{i + 3}{i} \right)C\\
\Downarrow \text{Comme c'est } & \text{un produit, certains éléments vont s'annuler,}\\
\text{puisqu'on va de } &\frac{4\rightarrow(n+3)}{1\rightarrow n} \text{on peut retirer tout ce qui est entre 4 et $n$}\Downarrow\\
&= \frac{(n+1)\phantom{a}(n+2)\phantom{a}(n+3)}{3 \times 2 \times 1}C\\
& a_n = \frac{1}{6}(n+1)(n+2)(n+3)C\\
\end{align*}
si $\tilde{a}_n = An^3 + Bn^2 + Cn$, alors $A=-1$, $B=-3$, $C=-2$\\\\
donc $a_n = \frac{1}{6}(n+1)(n+2)(n+3)C - (n^3 + 3n^2 + 2n)$.
\end{enumerate}

\end{document}
