\documentclass[fontsize=10pt]{article}
\usepackage[utf8]{inputenc}
\usepackage[T1]{fontenc}
\usepackage{graphicx} % handles figures
\usepackage[fleqn]{mathtools}
\usepackage{amsmath}
\usepackage{hyperref}


%Insertion de tout un tas de librairie qui nous seront probablement inutiles pour la pluspart mais it's always good to have them
\title{\textbf{How to latex}\\ il faut regarder le code ;)}
\author{Beltus Marcel}
\date{}
\begin{document}
\maketitle % fais le titre écris plus haut

\section{Crée une section}
\begin{enumerate} % commence une liste énumerée
\item item 1% ajoute un item dans la liste
\item item 2 % ajoute une deuxième item
\end{enumerate} % commence une liste énumerée 
\subsection{Crée une sous section}
\begin{itemize} % commence une liste avec des points
\item item 1% ajoute un item dans la liste
\item item 2 % ajoute une deuxième item
\end{itemize} % commence une liste énumerée 
Un passage à la ligne s'indique soit par espacer de deux lignes les lignes entre lesquels 

Nous voulons metre un espace. soit en mettant un double slash en fin de phrase\\

Comme ici
\section{Les maths}
pour écrire des maths en latex il faut rentrer dans un environnement mathématique dans lequel certains fonctions comme la création de racine carrées. exemple : $\sqrt{\pi}$\\
En regardant le code vous pouvez remarquer que pour afficher des objets mathématiques nous appelons des fonctions.(enfin ici ce sont plutot des macros mais chuut ;) ). avec ces fonctions nous pouvons affiché tout un tas de choses\\
Voici donc une liste non exhaustive d'outils mathémqitue utils pour ce que nous comptons faire.\\
N'oubliez pas que tout ces éléments doivent être utilisés dans des environnements mathématique.
\begin{itemize}
\item \textbackslash pi \textbackslash Pi affichent pi en minuscule et en minuscule respectivement, cela marche avec toutes les lettres grecs $\pi \Pi$
\item \textbackslash sqrt{} pour afficher une racine $\sqrt{test}$ il faut entre crochet mettre la valeur que l'on veut mettre en racine.
\item{Mettre un chiffre en exposant/sous exposant se fait en utilise le signe \textasciicircum /\textunderscore  suivie du chiffre. si l'on veut mettre des nombres en exposant/sous exposant il faut les entourer dans des corhcets. \\Exemple : $3^{34} 89_l$}
\item les signes sommatoires, de produits et d'intégrations peuvenet aussi vous servir on utilise \textbackslash sum, \textbackslash prod, \textbackslash int. \\Exemple :  $\sum_{i}^j = \prod^{78}_m $ et $\int_7^9$

\subsection{Les matrics}
C'est un peu chiant à faire mais c'est utile.
$\begin{bmatrix}
	b  \\
	a-b
	\end{bmatrix}$
$\begin{matrix}
	b  \\
	a-b
	\end{matrix}$
	$\begin{pmatrix}
	b & 9\\
	a & -b\\
	8  & 9 
	\end{pmatrix}$
	les collonnes sont séparées par des \& et les lignes par des \textbackslash\\\\
	la création se fait comme pour une liste mais en environnement mathématique. sauf qu'à la palce de itemize ou enumerate on a bmatrix, matrix, pmatrix et pleins d'autres que vous pouvez d'autres exemples ici \href{https://fr.wikibooks.org/wiki/LaTeX/%C3%89crire_des_math%C3%A9matiques}{ici}
	
\subsection{Les environnements mathématiques}
Vous connaisser déja l'environenemnt qui commence et se termine par un  \textdollar ce dernier permet de faire des environnements mathématiques inlines. si nous utilisons 2 signes  \textdollar au début et à la fin nous aurons un environnement mathématique centré sur une nouvelle ligne $$ x + y =\text{regardez ici}$$ . vous pouvez aussi voir dans cette exemple que j'ai écrit du texte dans l'environnement mathématique, cela se fait par la simple utilisaiton de la fonction  \textbackslash text{}\\

Il y a un dernier environnement dont j'aimerais vous parler le \emph{align} c'est un environnemnet mathématique dans lequel les différentes équations sont alignés par rapport à des éléments que l'utilisateur indique . On désigne quel élément il faut aligné en le faisant précédé de 1 ou 2 \& 

exmeple : 
\begin{align*}
a^2 &= b^2+c^2\\
a^2 &= 89 + 78\\
a &= \sqrt{89+78}\\
a &= 12.9228479833
\end{align*}
ici l'alignement se fait sur le signe =. on peut faire plusieurs alignemnt dans la même ligne mais il ne faut pas oublier d'avoir le même nomre de \& par ligne. Car l'alignement se fait en divisant l'espace en colonnes et les colonnes sont délimités par ces \&. Il est donc évident qu'il faut le même nombre de colonne à chaque ligne.\\
Dernière chose le * dans align* indique juste que les lignes ne doivent pas être numérotées. vous n'aurez probablement pas beosins d'utilisé une version numéroté mais si vous en avez besoins n'hésitez pas à me contacter.


\end{itemize}
\end{document}
