\documentclass[fontsize=10pt]{article}
\usepackage[utf8]{inputenc}
\usepackage[T1]{fontenc}
\usepackage{graphicx} % handles figures
\usepackage[fleqn]{mathtools}
\usepackage{amsmath}
\usepackage{hyperref}
\usepackage{amssymb}
\usepackage{xcolor}

%Insertion de tout un tas de librairie qui nous seront probablement inutiles pour la pluspart mais it's always good to have them
\title{\textbf{Maths Discrètes}\\ Solutions TP 3}
\author{Beltus Marcel}
\date{}
\begin{document}
\maketitle % fais le titre écris plus haut


\section*{Exercice 1}
$s = 15$, \hspace{1cm}$d = 4$\\
$$\begin{pmatrix}
s + d -1\\
s
\end{pmatrix}
=
\begin{pmatrix}
18\\
 15
\end{pmatrix}$$
\section*{Exercice 2}
\begin{align*}
&x > 0 \rightarrow x \geq 1 \hspace{0.5cm} &;\hspace{0.5cm}&x' = x-1 \hspace{1cm}\text{ou} \hspace{0.5cm}&x = x'+1 \\
&y \geq 9 &; \hspace{0.5cm}&y' = y-9 & y = y'+9\\
&z>-1 \rightarrow z\geq -1&; \hspace{0.5cm}&z' = z+1 & z = z'-1\\
&t \geq 0 &; \hspace{0.5cm}&t' = t & t = t'\\
&y > 10 \rightarrow u \geq 11 &; \hspace{0.5cm}&u' = u-11 & u = u' +11
\end{align*}
\section*{Exercice 3}
\begin{enumerate}
\item $s \leq 6 $ et $s \in \mathbb{N}$ \hspace{0.5cm}, \hspace{0.5cm} d = 4
\begin{align*}
&\underset{n=0}{\overset{6}{\sum}}\begin{pmatrix}
n + 4 -1\\
4-1
\end{pmatrix} &&=
\underset{n=0}{\overset{6}{\sum}}\begin{pmatrix}
n + 3\\
3
\end{pmatrix}\\
&&&=
\underset{n=0}{\overset{6}{\sum}}\begin{pmatrix}
n + 3\\
0+3
\end{pmatrix}\\
&&&=\underset{n=0}{\overset{6}{\sum}}\begin{pmatrix}
6+1+3\\
0+1+3
\end{pmatrix}\\&&&= \begin{pmatrix} 10 \\ 4\end{pmatrix} = 210
\end{align*}
\item$ 0 < s \leq 6$ et $s \in \mathbb{Z}_+$, $d=4$
\begin{align*}
\underset{n=1}{\overset{6}{\sum}}
\begin{pmatrix}
n+4-1\\
4-1
\end{pmatrix} &=\underset{n=1}{\overset{6}{\sum}}
\begin{pmatrix}
n+3\\
4-1
\end{pmatrix}\\
&=\underset{n=0}{\overset{6}{\sum}}
\begin{pmatrix}
n+3\\
0+3
\end{pmatrix}
-
\begin{pmatrix}
3\\
3
\end{pmatrix}\\
&=
\begin{pmatrix}
10\\
4
\end{pmatrix}-1\\
&=209 \text{\textcolor{red}{ J'ai pas compris tes annotation}}
\end{align*}

\item
\begin{alignat*}{4}
 &x \geq 3 &&; x =x' +3\\
&y \geq -1 &&; y = y' -1\\
&z \geq 1 &&; z = z' +1\\
&t \geq -2 &&; t = t'-2
\end{alignat*}
$\Rightarrow$ $(x'+3)+(y'-1)+(z'+1)+(t'-2) \leq 6$\\
$\Leftrightarrow$ $x'+y'+z'+t' \leq 5$ \\
$\Rightarrow$ $s \leq 5 \text{ et } s \in \mathbb{Z}, d=4$ \hspace{3cm} $\in \mathbb{Z}$ parce que $x,y,z,t\geq 0$\\
\begin{alignat*}{4}
\underset{n=0}{\overset{5}{\sum}}
\begin{pmatrix}
n + 4 -1\\
4-1
\end{pmatrix}
&= \underset{n=0}{\overset{5}{\sum}}
\begin{pmatrix}
n+3\\
3
\end{pmatrix}\\
&= 
\begin{pmatrix}
9\\
4
\end{pmatrix} = 126
\end{alignat*}
\end{enumerate}
\section*{Exercice 4}
$$x+y+z = 415 -t$$
$$(415-t)+u = 273$$
$$t-u = 142$$
$$\text{si on conniat $u$ on connait $t$}$$
$$\text{(et $u < t$ donc $\mathbb{Z}_+$)}$$
$$ (x'+1)+(y'+1)+(z'+1)+(u'+1)=273$$
$$ x'+y'+z'+u' = 269$$
$$\begin{pmatrix}
269+4-1\\
4-1
\end{pmatrix} = 
\begin{pmatrix}
272\\
3
\end{pmatrix}
$$
\section*{Exercice 5}
$$ (x'+1)+(y'+1)+(z'+1)+(t'+1) \leq 99$$
$$ x'+y'+z'+t' \leq 95$$
$$ \underset{n=0}{\overset{95}{\sum}}
\begin{pmatrix}
n+3\\
3
\end{pmatrix} = \begin{pmatrix}
98\\
3
\end{pmatrix}$$



\section*{Exercice 6}
Avec ces 12 lettres nous pouvons composer $\dfrac{12!}{2!2!2!2!3!1!}$\\
Il y a 13 emplacements pour U (avant, enter, après)\\
On a 9 U, donc $\begin{pmatrix}
13\\
9
\end{pmatrix}$ possibilités

ou tous les mots - chaque exceptoin (2u,3u,$\dots$cotes à cote). \textcolor{red}{ ça m'a pas l'air très claire.}
\section*{Exercice 7}
16
\section*{Exercice 8}
Contradiction\\
$\rightarrow$ max$_n$ = identiques \textcolor{red}{ J'ai pt mal lu ça me smeble ambigu}\\  
$\rightarrow$ max$_n$ = différents\\
pour chaque elem $\in n$, au plus $n$ élements.
alors $n^2$ Objetcs
si $n^2+1$ alors soit $n+1$ éléments soit $n+1$ types d'un type
\section*{Exercice 9}
\begin{enumerate}
\item $8^{16}$
\item $8^8$
\item $8^{11}$
\item $7^{14}+7^{15}+7^{16}$
\item $7^{11}+7^{10}+7^{9}$
\end{enumerate}
\section*{Exercice 10}
Contradiction, Ramsay number $R_{3\cdot3}$


















































\end{document}
