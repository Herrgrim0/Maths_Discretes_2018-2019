\documentclass[fontsize=10pt]{article}
\usepackage[utf8]{inputenc}
\usepackage[T1]{fontenc}
\usepackage{graphicx} % handles figures
\usepackage[fleqn]{mathtools}
\usepackage{amsmath}
\usepackage{hyperref}
\usepackage{amssymb}

%Insertion de tout un tas de librairie qui nous seront probablement inutiles pour la pluspart mais it's always good to have them
\title{\textbf{Maths Discrètes }\\Solutions TP 2 }
\date{}
\begin{document}
\maketitle % fais le titre écris plus haut


\section*{Exercice 1}
\begin{enumerate}
\item $\begin{pmatrix}
7 \\ 5
\end{pmatrix} = 21$
\item $\begin{pmatrix} 
10\\
5
\end{pmatrix}
\begin{pmatrix} 
5\\
3
\end{pmatrix}
\begin{pmatrix} 
2\\
2
\end{pmatrix} = 2520$
\item $ 
\begin{pmatrix} 
4\\
1
\end{pmatrix} = 4$ ou $
\begin{pmatrix} 
4\\
3
\end{pmatrix} =4$ puisque 4 emplacements et 1 pot d econfiture, 3 pots de miel
\end{enumerate}
\section*{Exercice 2}
\begin{enumerate}
\item $\begin{pmatrix} 
4\\
1
\end{pmatrix}
\begin{pmatrix} 
10\\
4
\end{pmatrix} = 840$, choix de la couleur puis des cartes
\item$\begin{pmatrix} 
4\\
2
\end{pmatrix}
\begin{pmatrix} 
10\\
2
\end{pmatrix}
\begin{pmatrix} 
10\\
2
\end{pmatrix} + 
\begin{pmatrix} 
4\\
1
\end{pmatrix}
\begin{pmatrix} 
10\\
3
\end{pmatrix}
\begin{pmatrix} 
3\\
1
\end{pmatrix}
\begin{pmatrix} 
10\\
1
\end{pmatrix}$, choix des cartes pour une main de 2 cartes de chaque couleur, plus choix des cartes pour une main de 3 cartes d'une couleur, 1 de l'autre
\item $
\begin{pmatrix} 
4\\
3
\end{pmatrix}
\begin{pmatrix} 
3\\
1
\end{pmatrix}
\begin{pmatrix} 
10\\
2
\end{pmatrix}
\begin{pmatrix} 
10\\
1
\end{pmatrix}
\begin{pmatrix} 
10\\
1
\end{pmatrix}$, choix des couleurs puis de celle sur 2 cartes, puis choix des cartes
\item $10^4$ ou $
\begin{pmatrix} 
4\\
4
\end{pmatrix}
\begin{pmatrix} 
10\\
1
\end{pmatrix}
\begin{pmatrix} 
10\\
1
\end{pmatrix}
\begin{pmatrix} 
10\\
1
\end{pmatrix}
\begin{pmatrix} 
10\\
1
\end{pmatrix}$
\end{enumerate}

\section*{Exercice 3}
\begin{enumerate}
\item
$\begin{pmatrix} 
12\\
3
\end{pmatrix}
\begin{pmatrix} 
9\\
3
\end{pmatrix}
\begin{pmatrix} 
6\\
3
\end{pmatrix}
\begin{pmatrix} 
3\\
3
\end{pmatrix}$
\item $\dfrac{
\begin{pmatrix} 
12\\
3
\end{pmatrix}
\begin{pmatrix} 
9\\
3
\end{pmatrix}
\begin{pmatrix} 
6\\
3
\end{pmatrix}
\begin{pmatrix} 
3\\
3
\end{pmatrix}}{4!}$, pour éviter les groupes répétés en ordre différent


\end{enumerate}
\section*{Exercice 4}
\emph{Binome de Newton},  
\begin{align*}
(1+1)^n &=\underset{k=0}{\overset{n}{\sum}} 
\begin{pmatrix} 
n\\
k
\end{pmatrix}
\begin{pmatrix} 
1^k
\end{pmatrix}
\begin{pmatrix} 
1^{x-k}
\end{pmatrix}\\
2^n&=\underset{k=0}{\overset{n}{\sum}}
\begin{pmatrix} 
n\\
k
\end{pmatrix}
\end{align*}
Nombre de représentations de $\{ 1,\dots,n\}$ en deux sous-ensembles
\begin{itemize}
\item chaque élément n pris ou non pour le $1^{\text{er}}$ sous-ensemble, $2^n$
\item chaque sous-ensemble, de taille k, $
\begin{pmatrix} 
n\\
k
\end{pmatrix}$
\end{itemize}
\section*{Exercice 5}
par le binome de Newton, si $x=-1$ et $y=1$, alors
$$ \underset{k=0}{\overset{n}{\sum}}
\begin{pmatrix} 
n\\
k
\end{pmatrix}
\begin{pmatrix} 
-1
\end{pmatrix}^k
\begin{pmatrix} 
1
\end{pmatrix}^{n-k} = (-1+1)^n$$
$$ \underset{k=0}{\overset{n}{\sum}}
\begin{pmatrix} 
n\\
k
\end{pmatrix}
(-1)^k = 0
$$
\section*{Exercice 6}
$$ (x+y)^{n}
= \underset{k=0}{\overset{n}{\sum}}
\begin{pmatrix} 
n\\
k
\end{pmatrix}x^{k}\phantom{a}y^{n-k} $$
$$ n(x+y)^{n-1}
= \underset{k=0}{\overset{n}{\sum}}
\begin{pmatrix} 
n\\
k
\end{pmatrix}k\phantom{a}x^{k-1}\phantom{a}y^{n-k} $$
$$ n(x+y)^{n-1}
= \underset{k=1}{\overset{n}{\sum}}
\begin{pmatrix} 
n\\
k
\end{pmatrix}k\phantom{a}x^{k-1}\phantom{a}y^{n-k} $$
$$\Downarrow \text{comme } \begin{pmatrix} 
n\\
k
\end{pmatrix}k
= n\begin{pmatrix} 
n-1\\
k-1
\end{pmatrix} \Downarrow$$
$$ n(x+y)^{n-1}
= \underset{k=1}{\overset{n}{\sum}}
n\begin{pmatrix} 
n-1\\
k-1
\end{pmatrix}x^{k-1}\phantom{a}y^{n-k}, $$
$$ n(x+y)^{n-1}
= \phantom{a}n
\underset{k=0}{\overset{n-1}{\sum}}
\begin{pmatrix} 
n-1\\
k
\end{pmatrix}x^{k}\phantom{a}y^{n-1-k} $$
$$\Downarrow \text{si on prend $x=y=1$, alors:} \Downarrow$$
$$ n2^{n-1}
= \phantom{a}n
\underset{k=0}{\overset{n-1}{\sum}}
\begin{pmatrix} 
n-1\\
k
\end{pmatrix} $$
$$ n2^{n-1}
= \underset{k=0}{\overset{n}{\sum}}
\begin{pmatrix} 
n\\
k
\end{pmatrix}k $$
\section*{Exercice 7}
$\overset{n}{\underset{k \text{ pair} = 0}{\sum}}
\begin{pmatrix} 
n\\
k
\end{pmatrix}$; pour $(-1+1)^n$;
\begin{align*}
\overset{n}{\underset{k= 0}{\sum}}
\begin{pmatrix} 
n\\
k
\end{pmatrix}(-1)^k(1)^{n-k} &= 0\\
\overset{n}{\underset{k= 0}{\sum}}
\begin{pmatrix} 
n\\
k
\end{pmatrix}(-1)^k &=0\\
\overset{n}{\underset{k \text{ pair}}{\sum}}
\begin{pmatrix} 
n\\
k
\end{pmatrix}
-
\overset{n}{\underset{k\text{ impair}}{\sum}}
\begin{pmatrix} 
n\\
k
\end{pmatrix} &=0\\
\overset{n}{\underset{k \text{ pair}}{\sum}}
\begin{pmatrix} 
n\\
k
\end{pmatrix}
&=
\overset{n}{\underset{k\text{ impair}}{\sum}}
\begin{pmatrix} 
n\\
k
\end{pmatrix}
\end{align*}
\section*{Exercice 8}
Dans le triangle de Pascal\\
$\Rightarrow
\begin{pmatrix} 
n-1\\
k-1
\end{pmatrix}
\begin{pmatrix} 
n-1\\
k
\end{pmatrix}
\begin{pmatrix} 
n\\
k-1
\end{pmatrix}
\begin{pmatrix} 
n\\
k
\end{pmatrix}
\begin{pmatrix} 
n\\
k+1
\end{pmatrix}
\begin{pmatrix} 
n+1\\
k
\end{pmatrix}
\begin{pmatrix} 
n+1\\
k+1
\end{pmatrix}$\\
$=\dfrac{(n-1)!}{(k-1)!(n-1-k+1)!} \cdot \dfrac{(n-1)!}{k!(n-1-k)!}\dots$\\\\
$=\dfrac{(n-1)!^2\phantom{a}n!^2\phantom{a} (n+1)!^2}{(k-1)!^2(n-k)!^2(k+1)!^2(n-k-1)!^2k!^2(n-k+1)!^2}$\\\\
$=
\begin{pmatrix} 
n-1\\
k-1
\end{pmatrix}^2
\begin{pmatrix} 
n\\
k+1
\end{pmatrix}^2
\begin{pmatrix} 
n+1\\
k
\end{pmatrix}^2$
\section*{Exercice 9}
\begin{align*}
\overset{n}{\underset{k= 0}{\sum}}
\phantom{a}k
\begin{pmatrix} 
n\\
k
\end{pmatrix}
& = \overset{n}{\underset{k= 0}{\sum}}
\phantom{a}k\dfrac{n!}{k!\phantom{a}(n-k)!}\\
&\Downarrow\text{si $k=0$, tout es à 0. $k/k!$ $\rightarrow 1/(k-1$)}\Downarrow\\
& = \overset{n}{\underset{k= 1}{\sum}}
\phantom{a}\dfrac{n!}{(k-1)!\phantom{a}(n-k)!}\\
& = n\overset{n}{\underset{k= 1}{\sum}}
\phantom{a}\dfrac{(n-1)!}{(k-1)!\phantom{a}(n-k)!}\\
&\Downarrow\text{où L = k =1}\Downarrow\\
& = n\overset{n-1}{\underset{L= 0}{\sum}}
\phantom{a}\dfrac{(n-1)!}{L!\phantom{a}(n-1-L)!}\\
&= n2^{n-1}
\end{align*}






\section*{Exercice 10}
\begin{align*}
\overset{n}{\underset{k= 0}{\sum}}
\phantom{a}\frac{1}{k+1}
\begin{pmatrix}
n\\
k
\end{pmatrix}
& = \frac{1}{n+1}\phantom{a}
\overset{n}{\underset{k= 0}{\sum}}
\phantom{a}\frac{n!\phantom{a}(n+1)}{(k+1)\phantom{a}k!\phantom{a}(n-k)!}\\
& = \frac{1}{n+1}\phantom{a}
\overset{n}{\underset{k= 0}{\sum}}
\begin{pmatrix}
n+1\\
k+1
\end{pmatrix}\\
& \Downarrow\text{pour arriver à }
\overset{n+1}{\underset{k= 0}{\sum}}
\begin{pmatrix}
n+1\\
k
\end{pmatrix} = 2^{n+1} \text{, il faut pour $k=0$}\Downarrow\\
& = \frac{1}{n+1}\phantom{a}
\left( \overset{n}{\underset{k= 0}{\sum}}
\begin{pmatrix}
n+1\\
k+1
\end{pmatrix} + 
\begin{pmatrix}
n+1\\
0
\end{pmatrix} -
\begin{pmatrix}
n+1\\
0
\end{pmatrix} \right)\\
& = \frac{1}{n+1}(2^{n+1}-1)
\end{align*}


\end{document}
